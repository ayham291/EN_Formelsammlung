\clearpage
\section{Generator}
\textbf{Induktionsgesetz}
\begin{align*}
    U_{ieff} &= 4,44 \cdot N \cdot f \cdot B_= \cdot \hat{A}_{Fe}
\end{align*}

\textbf{Polpaarzahl}
\begin{align*}
    n &=\frac{f}{p}= \dfrac{3000}{p} \left[ \frac{1}{min}\right] \qquad \textnormal{für f = 50Hz}
    \end{align*}

\subsection{Ersatzschaltbild (ESB)}
Vernachlässigung von $R_S = 0$
\begin{align*}
    \ul{I} &= I_w +jI_b \qquad I_w = I \cdot \cos(\varphi) \quad I_b = \ul{I}\cdot (\pm sin(\varphi))\\
    \ul{U}_p &= \ul{U} +jX_s \cdot \ul{I} \qquad
    U_p = \sqrt{(U+X\cdot I_B)^2 \cdot (X\cdot I_W)^2}
\end{align*}
$\ul{U}, \ul{I}$: Klemmen \quad $U_p$: Polrad (Quelle) \quad $X_s$: Statorreaktanz \\
$\varphi$: $\sphericalangle (U, I)$ \qquad $\vartheta$:$\sphericalangle (U, U_p)$

\subsection{Allein-/Inselbetrieb}
Belastungskennline ($U_p=$ const.):
\begin{align*}
    1 &= \left( \dfrac{U}{U_p} \right)^2 - 2 \cdot \left( \dfrac{U}{U_p}\right) \cdot \left( \dfrac{I}{I_k}\right) \cdot (\pm \sin(\varphi)) + \left(\dfrac{I}{I_k}\right)^2
\end{align*}
\subsubsection{Reine Wirkleistung}
$sin(\varphi) = 0 \Rightarrow$ Kreis
\begin{align*}
    \dfrac{U}{U_p} &= \sqrt{1- \left(\dfrac{I}{I_k}\right)^2} \qquad I_k = \frac{U_p}{X_s}
\end{align*}

\subsubsection{Reine Blindleistung}
$    sin(\varphi) = \pm1 \Rightarrow$ Gerade
\begin{align*}
    \dfrac{U}{U_p} &= 1 \mp \left(\dfrac{I}{I_k}\right)
\end{align*}
$- :$ induktiv \qquad $+ :$kapazitiv \qquad
$I_k :$ Kurzschluss

\subsection{Netzbetrieb, Betrieb am starren Netz}
Regulierkennlinie ($U=$ const.):
\begin{align*}
    \dfrac{U_p}{U} &=  \sqrt{1+2\cdot \dfrac{I}{I_B} \cdot \sin(\varphi) + \left( \dfrac{I}{I_B}\right)^2} \qquad I_B = \frac{U}{X}
\end{align*}
\subsubsection{Reine Wirkleistung}
$sin(\varphi) = 0 \Rightarrow$ Kreis
\begin{align*}
    \dfrac{U_p}{U} &= \sqrt{1+ \left(\dfrac{I}{I_B}\right)^2}
\end{align*}

\subsubsection{Reine Blindleistung}
$    sin(\varphi) = \pm1 \Rightarrow$ Gerade
\begin{align*}
    \dfrac{U_p}{U} &= 1 \pm \left(\dfrac{I}{I_B}\right)
\end{align*}
$+ :$ induktiv \qquad $- :$kapazitiv \qquad $I_B :$ Bezugsstrom

\subsection{Leistung}
$P_{mech,zu} = P_{el,ab} \neq f(\vartheta) \rightarrow$ für $\vartheta < 90\degree$ stabil\\
\begin{align*}
    sin(\vartheta) = \frac{I_w}{U_p} &= \frac{P\cdot X}{3\cdot U \cdot U_p}  \qquad P = 3 \cdot \frac{U\cdot U_p}{X} \cdot sin(\vartheta)\\
    S_{Bez} = 3 \cdot \dfrac{U^2}{X} & \qquad P_{Kipp} = P_{max} = 3\cdot U \cdot \frac{U_p}{X}\\
    \dfrac{P}{S_{Bez}} &= \left(\dfrac{U_p}{U}\right) \cdot sin(\vartheta)\\
    \dfrac{Q}{S_{Bez}} &= \left[\left(\dfrac{U_p}{U}\right) \cdot cos(\vartheta)\right] -1\\
    % P_{Kipp} &= 3 \cdot \dfrac{U \cdot U_p}{X_S}
\end{align*}

\subsection{Regelung}
\begin{align*}
    \upsilon_{neu} &= sin^{-1}\left( \dfrac{P_{neu} \cdot X}{3 \cdot U \cdot U_{p,neu}}\right)\\
    Q_{neu} &= \left( \dfrac{3 \cdot U^2}{X_s}\right) \cdot \left( \left( \dfrac{U_{p,neu}}{U} cos( \upsilon_{neu})\right)-1\right)
\end{align*}

\textbf{Konstante Scheinleistung}\\
\indent $S = S_{max} =$ const\\
\indent $\ul{U}=$ const, $|\ul{I}| =$ const\\
\indent $I_w =$ var. $\rightarrow P_{zu}=$ var.\\
\indent $I_b(I_{err}) =$ var. $\rightarrow U_p =$ var.\\

\textbf{Regelung Polradspannung}\\
\indent $I_{err}(I_b) \sim U_p$ \\
\indent $P =$ const $\rightarrow I_w= $ const\\
\indent $U_p =$ var. $\rightarrow I_b/I_{err} = $ var. $\rightarrow Q=$ var.

\indent bei $1 \leq \frac{U_p}{U} \leq 2$ ergibt sich \\
\indent $-0,5 (kap.) \leq \frac{Q}{S_{Bez}} \leq +0,75(ind.)$\\

\textbf{Regelung Turbinenleistung}\\
\indent $U_p =$ const $\rightarrow I_b = f(I_w)$\\
\indent $P =$ var. $\rightarrow I_w= $ var. $\rightarrow Q =$ var.

\indent bei $0,9 \leq \frac{P}{S_{Bez}} \leq 1,75$ ergibt sich\\
\indent $0 \leq \frac{Q}{S_{Bez}} \leq +0,75(ind.)$

$U_p$, $ P$, $Q$ und $\upsilon$ sind abhängig voneinander. \\
Aus zwei Größen $\rightarrow$ die anderen Beiden\\

\textbf{Reine Blindleistung (Phasenschieberbetrieb)}\\
\indent $P = 0 \rightarrow I_w= 0$ \\
$U, U_p$ phasengleich\\
\indent $Q =$ var. $\rightarrow I_b = $ var. $\rightarrow U_p$ var.\\
Bei Leerlauf: $I_b = 0 \rightarrow U = U_p$
