\section{Generator}
\textbf{Induktionsgesetz}\\
\textbf{Polpaarzahl}
\begin{align*}
    U_{ieff} &= 4,44 \cdot N \cdot f \cdot B_= \cdot \hat{A}_{Fe} \\
    n &= \dfrac{3000}{p} \left[ \frac{1}{min}\right] \qquad \textnormal{für f = 50Hz}
\end{align*}

\subsection{Ersatzschaltbild (ESB)}
\begin{align*}
    \ul{U}_p &= \ul{U} + (R_S + jX_S) \cdot \ul{I} \qquad \textnormal{$R_S$ wird meist vernachl.}\\
    1 &= \left( \dfrac{U}{U_p} \right)^2 - 2 \cdot \left( \dfrac{U \cdot I}{U_p \cdot I_k}\right) sin(\varphi) + \left(\dfrac{I}{I_k}\right)^2
\end{align*}

\subsection{Alleinbetrieb (Inselbetrieb)}
\subsubsection{Reine Wirkleistung}
\begin{align*}
    cos(\varphi) &= 1\\
    1 &= \left( \dfrac{U}{U_p} \right)^2 + \left(\dfrac{I}{I_k}\right)^2
\end{align*}

\subsubsection{Reine Blindleistung}
\begin{align*}
    sin(\varphi) &= 1 \rightarrow (ind.) \qquad =-1 \rightarrow (kap.)\\
    \left( \dfrac{U}{U_p} \right) &= 1 \mp \left(\dfrac{I}{I_k}\right)
\end{align*}

\subsection{Leistung}
\begin{align*}
    S_{Bez} &= 3 \cdot \dfrac{U^2}{X_S}\\
    \dfrac{P}{S_{Bez}} &= \left(\dfrac{U_p}{U}\right) sin(\upsilon)\\
    P &= \dfrac{3 \cdot U \cdot U_p}{X_S} \cdot sin (\upsilon)\\
    P_{mech,zu} &= P_{el,ab} \neq f(\upsilon) \rightarrow \textnormal{für $\upsilon < 90\degree$ stabil}\\
    P_{Kipp} &= 3 \cdot \dfrac{U \cdot U_p}{X_S}\\
    \dfrac{Q}{S_{Bez}} &= \left(\left(\dfrac{U_p}{U}\right) cos(\upsilon)\right) -1    
\end{align*}

\subsection{Regelung}

\textbf{Konstate Scheinleistung}\\
\indent $S = S_{max} =$ const\\
\indent $\ul{U}=$ const, $|\ul{I}| =$ const\\
\indent $I_w =$ var. $\rightarrow P_{zu}=$ var.\\
\indent $I_b =$ var. $\rightarrow I_{Err}, U_p =$ var.\\

\textbf{Blindleistung}\\
\indent $P =$ const $\rightarrow I_w= $ const\\
\indent $Q =$ var. $\rightarrow I_b = $ var. $\rightarrow I_{Err}=$ var.\\
\indent bei $1 \leq \frac{U_p}{U} \leq 2$ ergibt sich \\
\indent $-0,5 (kap.) \leq \frac{Q}{S_{Bez}} \leq +0,75(ind.)$\\

\textbf{Turbinenventildrosselung}\\
\indent $P =$ var. $\rightarrow I_w= $ var. $\rightarrow Q =$ var.\\
\indent $U_p =$ const $\rightarrow I_b = f(I_w)$\\
\indent bei $0,9 \leq \frac{P}{S_{Bez}} \leq 1,75$ ergibt sich\\
\indent $0 \leq \frac{Q}{S_{Bez}} \leq +0,75(ind.)$\\

\begin{align*}
    \upsilon_{neu} &= sin^{-1}\left( \dfrac{P_{neu} \cdot X_S}{3 \cdot U \cdot U_p}\right)\\
    Q_{neu} &= \left( \dfrac{3 \cdot U^2}{X_s}\right) \cdot \left( \left( \dfrac{U_p}{U} cos( \upsilon_{neu})\right)-1\right)
\end{align*}

\textbf{Reiner Blindleistung (Phasenschieberbetrieb)}\\
\indent $P = 0 \rightarrow I_w= 0$ \\
\indent $Q =$ var. $\rightarrow I_b = $ var.
